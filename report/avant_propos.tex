\section*{Présentation}
Ce mémoire présente le travail réalisé pour l'obtention d'un Master en sciences informatiques à l'université de Genève.
%Todo organisation générale des chapitres du mémoire


\section*{Conventions typographiques}\label{title-avantpropos}

\noindent Ce document suit des règles typographiques afin d'en simplifier la lecture:

\begin{itemize}
\item l'\textit{italique} est employé pour les termes anglais ou sur lesquels l'attention est attirée;
\item une police de caractère à \texttt{chasse fixe} indique un extrait de code, une commande, un chemin ou la sortie standard d'une console.
\end{itemize}

\section*{Remerciements}
Je souhaite remercier chaleureusement Jonas Latt, pour son suivi régulier ainsi que sa patience tout au long de ce travail; Paul Albuquerque, pour son soutien; mes camarades et amis Thomas Bertrand et Jayro Aldaz, pour leur compagnie dans ces innombrables et interminables week-ends et soirées de travail à Battelle (puis chez Costa...) et sans qui je ne serai probablement pas arrivé jusqu'ici; Stéphane et mes anciens collègues assistants, pour ces trois belles années à l'hepia et leurs constants rappels qu'il serait bien d'enfin finir ce Master; Céline, pour son précieux soutien; ma mère pour la relecture de ce document; et finalement ma famille et mes proches.

%hepia CF