La simulation de fluide numérique a de nombreuses applications dans domaines aussi varié que la géologie pour l'étude des volcans \cite{brogi_lattice_2017}, du jeux-vidéo pour un rendu réaliste de l'eau, du médical pour modéliser l'écoulement du sang \cite{hirabayashi_lattice_2004}, de l'aviation pour étudier l'aérodynamisme d'un fuselage et en prédire le bruit \cite{lew_noise_2010}, et bien autres encore. C'est par conséquent un sujet qui intéresse de nombreux chercheurs.

La méthode de Lattice Boltzmann (\acf{LBM} en anglais) est un algorithme de simulation de fluide attractive pour sa disposition à modéliser des comportements de fluides complexes et à être parallélisé.

Palabos, un solveur en mécanique des fluides numérique basé sur \acs{LBM}, permet de distribuer la simulation sur un \textit{cluster} de \acs{CPU} pour accélérer les calculs. Cette parallélisation pourrait encore être poussée  davantage pour certaines parties du domaine en les calculant sur \acs{GPU}. En effet, ces derniers se prêtent particulièrement bien aux algorithmes parallèles.

C'est là l'objectif de ce travail qui propose une implémentation de \acs{LBM} sur \acs{GPU}, réalisée pour Palabos afin d'en améliorer les performances.

Après une vue d'ensemble des méthodes utilisées dans la simulation de fluide avec \acs{LBM}, ce document présente ensuite l'implémentation réalisée à travers ses deux principales phases de développement; la réalisation du code \acs{GPU} puis son intégration à Palabos.
Le document discute ensuite des performances mesurées pour estimer les capacités de l'implémentation, et propose finalement un modèle de performance pour les prédire.
