La simulation de fluide numérique a de nombreuses applications dans des domaines aussi variés que la géologie, pour l'étude des volcans \cite{brogi_lattice_2017}, du jeux-vidéo, pour un rendu réaliste de l'eau, du médical, pour modéliser l'écoulement du sang \cite{hirabayashi_lattice_2004}, de l'aviation. pour étudier l'aérodynamisme d'un fuselage et en prédire les sources de bruit \cite{lew_noise_2010}, et bien autres encore. C'est par conséquent un sujet qui intéresse de nombreux chercheurs.

La méthode de Lattice Boltzmann (\acf{LBM} en anglais) est un algorithme de simulation de fluide attractive pour sa disposition à modéliser des comportements de fluides complexes et à être parallélisé.

Palabos est un solveur en mécanique des fluides numérique basé sur \acs{LBM} qui permet de distribuer une simulation sur un \textit{cluster} de \acs{CPU} pour en accélérer les calculs. Ses performances pourrait être améliorée encore davantage avec l'utilisation de \acs{GPU} pour certaines parties du domaine. En effet, cette technologie, conçue pour réaliser des calculs parallèles à haute vitesse, se prête à la résolution des calculs que demande \acs{LBM}.

Pour y parvenir, ce travail propose une implémentation de \acs{LBM} sur \acs{GPU}, réalisée pour être intégré à Palabos sous la forme d'un module, nommé co-processeur ou accélérateur.

Ce document commence par présenter une vue d'ensemble des méthodes utilisées dans la simulation de fluide avec \acs{LBM}. Il détaille ensuite l'implémentation réalisée, à travers ses deux principales phases de développement: la réalisation du code \acs{GPU} puis son intégration à Palabos.
Un chapitre analyse par la suite les mesures de performances réalisées pour estimer les capacités de l'implémentation et en dériver un modèle de performance, et ainsi prédire son comportement des des configuration différentes. Ce modèle est présenté pour l'implémentation \acs{GPU} et esquissé pour l'approche hybride avec Palabos.
Cette thèse est finalement conclue avec un bilan des améliorations envisageables et du travail réalisé.
