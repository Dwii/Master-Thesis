\documentclass[a4paper,11pt]{article}
\advance\voffset by -20mm
\advance\hoffset by -15mm
\setlength{\textwidth}{165mm}
\setlength{\textheight}{240mm}

\usepackage{graphicx}
\usepackage{amsmath}
\usepackage{url}
\usepackage{bm}

%\newcommand{\vect}[1]{\vec{#1}}
\newcommand{\vect}[1]{\boldsymbol{#1}}
\newcommand{\vvect}[1]{\overrightarrow{#1}}
\newcommand{\tens}[1]{\boldsymbol{#1}}
\newcommand{\scalar}[2]{\vect{#1}\cdot\vect{#2}}
\newcommand{\nonbfscalar}[2]{{#1}\cdot{#2}}
\newcommand{\nnabla}{{\bf \nabla}}
\newcommand{\nablaone}{\nabla_{\hspace{-.2em}1}}
\newcommand{\nnablaone}{\nnabla_{\hspace{-.2em}1}}
\newcommand{\deltafieq}{\delta\hspace{-.1em}f_i^{eq}}
\newcommand{\divergence}[1]{\nnabla \cdot #1}
\newcommand{\contract}[2]{\tens{#1}:\tens{#2}}
%\newcommand{\laplacian}{\nabla^2}
\newcommand{\laplacian}{{\Delta}}
\newcommand{\hermitscalar}[2]{\left<#1|#2\right>}
\newcommand{\Mach}{\mathit{Ma}}
\newcommand{\LE}{\mathit{LE}}
\newcommand{\DP}{\mathit{DP}}
\newcommand{\trace}{\mathit{Tr}}

\newcommand*{\Scale}[2][4]{\scalebox{#1}{$#2$}}%
\newcommand{\tautensor}{\ensuremath{\tens{\tau}}}
\newcommand{\pitensor}{\ensuremath{\tens{\Pi}}}
\newcommand{\idtensor}{\ensuremath{\tens{I}}}

\title{Equations}

\begin{document}
\maketitle

\begin{equation}\label{eq:navier-stokes}
\partial_t\vect{u} + (\scalar{u}{\nnabla})\vect{u} = -\frac{1}{\rho_0}\nnabla p + \nu \laplacian\vect{u}.
\end{equation}

\begin{equation}\label{eq:poisson}
\scalar{\nnabla}{u} = 0
\end{equation}

\begin{equation}\label{eq:poisson}
\laplacian p = -\rho_0 (\nnabla\vect{u}):(\nnabla\vect{u})^T.
\end{equation}

\begin{equation}
\vect{u}(\vect{x},t)
\end{equation}
\begin{equation}
p(\vect{x},t)
\end{equation}
\begin{equation}
\nu
\end{equation}

\begin{equation}
\vec{u}^* = \frac{\vec{u}}{U}
\end{equation}

\begin{equation}
p^* = \frac{p}{\rho_0 U^2}
\end{equation}

\begin{equation}
\partial_t^* = \frac{L}{V} \partial_t
\end{equation}

\begin{equation}
\vec{\nabla}^* = L\vec{\nabla}
\end{equation}

\begin{equation}
\mathit{Re} = \frac{UL}{\nu}
\end{equation}


\begin{equation}\label{eq:navier-stokes}
\partial_t^*\vect{u^*} + (\scalar{u^*}{\nabla ^*})\vect{u^*} = -\nnabla p ^* + \frac{\nu}{UL} \laplacian\vect{u^*}.
\end{equation}

\begin{equation}
f_i^\mathrm{in}(\vect{x},t)
\end{equation}

\begin{equation}
f_i^\mathrm{out}(\vect{x},t)
\end{equation}

\begin{equation}
f^\mathrm{in}
\end{equation}

\begin{equation}
f^\mathrm{out}
\end{equation}

\begin{equation}
\rho(\vect{x},t) = \sum_{i=0}^{8} f_i^\mathrm{in}(\vect{x},t)
\end{equation}

\begin{equation}
\vect{u}(\vect{x},t) = \frac{1}{\rho(\vect{x},t)}\frac{\delta x}{\delta t}\sum_{i=0}^{8} \vect{v}_i f_i^\mathrm{in}(\vect{x},t)
\end{equation}

\begin{equation}
n_i^\mathrm{out}(\vect{x},t) = n_i^\mathrm{in}(\vect{x},t)+\Omega_i(\vect{x},t)
\end{equation}

\begin{equation}
f_i^\mathrm{in}(\vect{x},t) = f_i^\mathrm{out}(\vect{x}-\vect{v}_i\delta x, t-\delta t)
\end{equation}

\begin{equation}
n_i^\mathrm{in}(\vect{x},t) = n_i^\mathrm{out}(\vect{x}-\vect{v}_i\delta x, t-\delta t)
\end{equation}

\begin{equation}
p = c_s^2\rho
\end{equation}

\begin{equation}
c_s^2 = \frac{1}{3} \frac{\delta x^2}{\delta t^2}
\end{equation}

\begin{equation}
f_i^\mathrm{out} - f_i^\mathrm{in} = -\omega\left(f_i^\mathrm{in}-E(i,\rho,\vec{u})\right)
\end{equation}

\begin{equation}
E(i,\rho,\vect{u}) = \rho\, t_i\left(1+\frac{\nonbfscalar{\frac{\delta x}{\delta t}\vect{v}_i}{\vect{u}}}{c_s^2}+\frac{1}{2\,c_s^4}\left(\nonbfscalar{\Scale[1.0]{\frac{\delta x}{\delta t}}\vect{v}_i}{\vect{u}}\right)^2-\frac{1}{2\,c_s^2}\left|\vect{u}\right|^2\right)
\end{equation}

\begin{equation}
\nu = \delta t\,c_s^2\left(\frac{1}{\omega}-\frac{1}{2}\right)
\end{equation}

\begin{equation}
f_8^\mathrm{in}
\end{equation}

\begin{equation}
f_7^\mathrm{in}
\end{equation}

\begin{equation}
f_6^\mathrm{in}
\end{equation}

\begin{equation}
f_5^\mathrm{in}
\end{equation}

\begin{equation}
f_4^\mathrm{in}
\end{equation}

\begin{equation}
f_3^\mathrm{in}
\end{equation}

\begin{equation}
f_2^\mathrm{in}
\end{equation}

\begin{equation}
f_1^\mathrm{in}
\end{equation}

\begin{equation}
f_0^\mathrm{in}
\end{equation}

\begin{equation}
f_8^\mathrm{out}
\end{equation}

\begin{equation}
f_7^\mathrm{out}
\end{equation}

\begin{equation}
f_6^\mathrm{out}
\end{equation}

\begin{equation}
f_5^\mathrm{out}
\end{equation}

\begin{equation}
f_4^\mathrm{out}
\end{equation}

\begin{equation}
f_3^\mathrm{out}
\end{equation}

\begin{equation}
f_2^\mathrm{out}
\end{equation}

\begin{equation}
f_1^\mathrm{out}
\end{equation}

\begin{equation}
f_0^\mathrm{out}
\end{equation}

\begin{equation}
f_i^\mathrm{in}(\vect{x},t+1) = f_j^\mathrm{out}(\vect{x},t)
\end{equation}

\begin{equation}
v_i = -v_j
\end{equation}

\begin{equation}
f_0^\mathrm{in} = E(0,\rho\,\vect{u}) + (f_8^\mathrm{in}-E(8,\rho,\vect{u}))
\end{equation}

\begin{equation}
f_1^\mathrm{in} = E(1,\rho\,\vect{u}) + (f_7^\mathrm{in}-E(7,\rho,\vect{u}))
\end{equation}

\begin{equation}
f_2^\mathrm{in} = E(2,\rho\,\vect{u}) + (f_6^\mathrm{in}-E(6,\rho,\vect{u}))
\end{equation}

\end{document}

